\documentclass[english,man]{apa6}

\usepackage{amssymb,amsmath}
\usepackage{ifxetex,ifluatex}
\usepackage{fixltx2e} % provides \textsubscript
\ifnum 0\ifxetex 1\fi\ifluatex 1\fi=0 % if pdftex
  \usepackage[T1]{fontenc}
  \usepackage[utf8]{inputenc}
\else % if luatex or xelatex
  \ifxetex
    \usepackage{mathspec}
    \usepackage{xltxtra,xunicode}
  \else
    \usepackage{fontspec}
  \fi
  \defaultfontfeatures{Mapping=tex-text,Scale=MatchLowercase}
  \newcommand{\euro}{€}
\fi
% use upquote if available, for straight quotes in verbatim environments
\IfFileExists{upquote.sty}{\usepackage{upquote}}{}
% use microtype if available
\IfFileExists{microtype.sty}{\usepackage{microtype}}{}

% Table formatting
\usepackage{longtable, booktabs}
\usepackage{lscape}
% \usepackage[counterclockwise]{rotating}   % Landscape page setup for large tables
\usepackage{multirow}		% Table styling
\usepackage{tabularx}		% Control Column width
\usepackage[flushleft]{threeparttable}	% Allows for three part tables with a specified notes section
\usepackage{threeparttablex}            % Lets threeparttable work with longtable

% Create new environments so endfloat can handle them
% \newenvironment{ltable}
%   {\begin{landscape}\begin{center}\begin{threeparttable}}
%   {\end{threeparttable}\end{center}\end{landscape}}

\newenvironment{lltable}
  {\begin{landscape}\begin{center}\begin{ThreePartTable}}
  {\end{ThreePartTable}\end{center}\end{landscape}}

  \usepackage{ifthen} % Only add declarations when endfloat package is loaded
  \ifthenelse{\equal{\string man}{\string man}}{%
   \DeclareDelayedFloatFlavor{ThreePartTable}{table} % Make endfloat play with longtable
   % \DeclareDelayedFloatFlavor{ltable}{table} % Make endfloat play with lscape
   \DeclareDelayedFloatFlavor{lltable}{table} % Make endfloat play with lscape & longtable
  }{}%



% The following enables adjusting longtable caption width to table width
% Solution found at http://golatex.de/longtable-mit-caption-so-breit-wie-die-tabelle-t15767.html
\makeatletter
\newcommand\LastLTentrywidth{1em}
\newlength\longtablewidth
\setlength{\longtablewidth}{1in}
\newcommand\getlongtablewidth{%
 \begingroup
  \ifcsname LT@\roman{LT@tables}\endcsname
  \global\longtablewidth=0pt
  \renewcommand\LT@entry[2]{\global\advance\longtablewidth by ##2\relax\gdef\LastLTentrywidth{##2}}%
  \@nameuse{LT@\roman{LT@tables}}%
  \fi
\endgroup}


\ifxetex
  \usepackage[setpagesize=false, % page size defined by xetex
              unicode=false, % unicode breaks when used with xetex
              xetex]{hyperref}
\else
  \usepackage[unicode=true]{hyperref}
\fi
\hypersetup{breaklinks=true,
            pdfauthor={},
            pdftitle={The N400's 3 As: Association, Automaticity, Attenuation (and Some Semantics Too)},
            colorlinks=true,
            citecolor=blue,
            urlcolor=blue,
            linkcolor=black,
            pdfborder={0 0 0}}
\urlstyle{same}  % don't use monospace font for urls

\setlength{\parindent}{0pt}
%\setlength{\parskip}{0pt plus 0pt minus 0pt}

\setlength{\emergencystretch}{3em}  % prevent overfull lines

\ifxetex
  \usepackage{polyglossia}
  \setmainlanguage{}
\else
  \usepackage[english]{babel}
\fi

% Manuscript styling
\captionsetup{font=singlespacing,justification=justified}
\usepackage{csquotes}
\usepackage{upgreek}

 % Line numbering
  \usepackage{lineno}
  \linenumbers


\usepackage{tikz} % Variable definition to generate author note

% fix for \tightlist problem in pandoc 1.14
\providecommand{\tightlist}{%
  \setlength{\itemsep}{0pt}\setlength{\parskip}{0pt}}

% Essential manuscript parts
  \title{The N400's 3 As: Association, Automaticity, Attenuation (and Some
Semantics Too)}

  \shorttitle{N400 Association Attenuation}


  \author{Erin M. Buchanan\textsuperscript{1}, John E. Scofield\textsuperscript{2}, Nathan Nunley, \&\textsuperscript{3}}

  \def\affdep{{"", "", "", ""}}%
  \def\affcity{{"", "", "", ""}}%

  \affiliation{
    \vspace{0.5cm}
          \textsuperscript{1} Missouri State University\\
          \textsuperscript{2} University of Missouri\\
          \textsuperscript{3} University of Mississippi  }

  \authornote{
    \newcounter{author}
    Erin M. Buchanan, Department of Psychology, Missouri State University;
    John E. Scofield, Department of Psychology, University of Missouri,
    Columbia, MO, 65211; Nathan Nunley, University of Mississippi, P.O. Box
    1848, University, MS, 28677.

                      Correspondence concerning this article should be addressed to Erin M. Buchanan, 901 S National, Springfield, MO, 65897. E-mail: \href{mailto:erinbuchanan@missouristate.edu}{\nolinkurl{erinbuchanan@missouristate.edu}}
                                              }


  \abstract{The N400 waveform carries new insight into the nature of linguistic
processing and may shed light into the automaticity of priming word
relationships. We investigated semantic and associative word pairs in
classic lexical decision and letter search tasks to examine their
differences in cognitive processing. Normed database information was
used to create orthogonal semantic and associative word relationships to
clearly define N400 waveforms and priming for these pairs. Participants
showed N400 reduction for related word pairs, both semantic and
associative, in comparison to unrelated word pairs. This finding was
consistent across both lexical decision and letter search tasks,
indicating automatic access for both types of relatedness. Non-word
pairs showed N400 waveforms that resembled unrelated word pairs,
indicating the controlled examination of non-advantageous words.
Response latency data nearly mirrored the EEG finding. Priming was found
for semantic and associative word relationships, while non-word pairs
were generally slower than unrelated word pairs.}
  \keywords{association, semantics, priming, N400, EEG, lexical decision, letter
search \\

    \indent Word count: X
  }





\usepackage{amsthm}
\newtheorem{theorem}{Theorem}
\newtheorem{lemma}{Lemma}
\theoremstyle{definition}
\newtheorem{definition}{Definition}
\newtheorem{corollary}{Corollary}
\newtheorem{proposition}{Proposition}
\theoremstyle{definition}
\newtheorem{example}{Example}
\theoremstyle{definition}
\newtheorem{exercise}{Exercise}
\theoremstyle{remark}
\newtheorem*{remark}{Remark}
\newtheorem*{solution}{Solution}
\begin{document}

\maketitle

\setcounter{secnumdepth}{0}



Semantic facilitation through priming occurs when a related cue word
speeds the processing of a following target word ({\textbf{???}}). For
example, if a person is reading about a yacht race, the word boat is
easier to process because of previous activation in semantic memory.
Research suggests that priming transpires by both automatic and
controlled processes. The automatic model proposes that related words
are linked in the brain due to overlapping features ({\textbf{???}}).
Target words are activated without conscious control due to automatic
spreading activation within related cognitive networks. Lexical and
feature networks are thought to be stored separately, so that semantic
priming is the activation from the feature network feeding back into the
lexical level ({\textbf{???}}). The overlap of a second word's semantic
relatedness makes word recognition easier because it, in essence, has
already been processed. The controlled process model proposes that
people actively use cognitive strategies to connect related words
together. ({\textbf{???}}) describes both expectancy generation and post
lexical matching as ways that target word processing may be speeded. In
expectancy generation, people consciously attempt to predict the words
and ideas that will appear next, especially in sentences. Whereas in
post lexical matching, people delay processing of the second target word
so that it can be compared to the cue word for evaluation. In both
cases, the target word is quickened by its relationship to the cue word.

Traditionally, priming has been tested with a simple word or nonword
decision called a lexical decision task. Participants are shown a cue or
priming word, followed by a related or unrelated target word for the
word/nonword judgment. Priming occurs when the judgment for the target
is speeded for related pairs over unrelated pairs. Lexical decision
tasks have been criticized for their inability to distinguish between
automatic and controlled processing, so both single presentation lexical
decision tasks and masked priming manipulations have been introduced to
negate controlled processing ({\textbf{???}}). In a single lexical
decision task, participants assess both the cue and target word so that
they are not as overtly paired together. Experimenters might also mask
or distort the cue word, so that participants do not believe they can
perceive the cue word. Even though words are garbled, word perception
occurs at a subliminal level and often facilitates the target word with
automatic activation.

\subsection{Priming in the Brain}\label{priming-in-the-brain}

Event related potentials (ERPs) are used to distinguish both the nature
of priming and the automaticity of priming. The use of ERPs is
advantageous, measuring brain activity per an electroencephalogram (EEG)
with good temporal resolution, and is thought to be a sensitive measure
of real-time language processing ({\textbf{???}}). The N400 is a
negative waveform that occurs 400 msec after the participant is
presented with a stimulus ({\textbf{???}}). The N400 has been described
as a \enquote{contextual integration process}, in which meanings of
words are integrated and functions, bridging together sensory
information and meaningful representations ({\textbf{???}}). The
amplitude of the N400 is sensitive to contextual word presentations,
varying systematically with semantic processing. This change justifies
the use of the N400 as an appropriate dependent measure for lexical
decision tasks. When presented with related words, there is an
attenuation of the N400, meaning a more positive waveform when compared
to unrelated word presentation. This difference in waveforms indicates a
lessened contextual integration process because word meanings are
already activated.

Multiple theories of the N400, however, have been proposed and debated
on what explicitly the N400 indexes. On one hand, processes associated
with the N400 are believed to occur post-word recognition.
({\textbf{???}}) examined a lexical decision task paired with masked
priming. No differences were found in the N400 wave between related and
unrelated words in the masked prime condition. ({\textbf{???}})
concluded that this finding indicated that semantic activation was a
controlled process, because attenuation only occurred when words were
known. Thus, an ?integrating? process transpires with semantic
information from of multi-word characteristic representations
({\textbf{???}}, {\textbf{???}}). This condition supposedly rules out
automatic processes, because the masked prime condition only allowed
automatic processes to take place. Masked priming did not allow the
participants to consciously name the prime words they had seen; thus,
they were not able to purposefully employ conscious cognitive strategies
in processing these words. However, ({\textbf{???}}) have found that
with shorter stimulus onset asynchronies (SOAs), this effect of masked
priming disappears. SOAs are the time interval between the prime word
presentation and the target word appearance. Short SOAs are thought to
only allow for automatic processing because the controlled, attention
based processing has not had time yet to occur. Their study showed the
masked primes long enough to enhance priming, while remaining
imperceptible. With these modifications, ({\textbf{???}}) found equal
N400 attenuation for the masked and unmasked primes. This result would
indicate that automatic activation was taking place, as the masked prime
condition did not allow controlled processes to take place.
({\textbf{???}}) has found similar results in the N400 using different
masking levels, which kept judgment ability of prime words below chance.

A separate theory suggests that N400 effects are seen pre-word
recognition. The N400 was found to be sensitive to pseudo- or non-words,
even when absent a resemblance to real word counterparts.
({\textbf{???}}) explain that this result could imply processes that
precede word recognition, such as orthographic or phonological analysis.
More recently, ({\textbf{???}}) suggested that the N400 indexes access
to semantic memory. Meaningful stimuli representing a multitude of
modalities indicates a sensitivity with attention, albeit still can
occur in its absence. Processing from modalities can integrate, yielding
different meanings from different contexts, respectively
({\textbf{???}}). Regardless of competing aspects as to what the N400 is
estimated to index, vital insights have been made crossing different
cognitive domains, with the N400 illuminating aspects originating from
these different domains ({\textbf{???}}).

({\textbf{???}}) used the attention blink rapid serial visual
presentation (RSVP) paradigm, in which participants identified target
words within a stream of distractor words presented in a different
color. By selecting items via specifying the row and column within a
matrix, participants identified the target word they had previously
seen. These studies compare to masked priming, and show automatic
activation of semantic information even when targets were missed
({\textbf{???}}).

Letter search tasks also reduce semantic priming by focusing attention
on the lexical level instead of a feature meaning level
({\textbf{???}}). In this task, participants are asked to determine if
cue and target words contain a specific letter presented.
({\textbf{???}}) stipulate that this eliminated or reduced priming
indicates non-automatic semantic priming. However, it is also important
to note that ({\textbf{???}}) did yield evidence that letter search
primes produced semantic priming for low-frequency targets, albeit not
for high-frequency targets. In ({\textbf{???}}) letter search and
lexical decision combined study, they found that the letter search task
eliminated semantic priming when compared to unrelated word pairs and
the lexical decision task. Yet, ({\textbf{???}}) found ERP evidence for
semantic processing of the prime word during letter search tasks with
the attenuation of the N400.

\subsection{Association}\label{association}

From a theoretical standpoint, the relation between associative and
semantic processing follows a deep line of research. Associative word
pairs are words that are linked in one's memory by contextual
relationships, such as basket and picnic ({\textbf{???}}). Another
example would be a word pair like alien and predator, which would be
associatively linked for Americans due to the movies and popular
culture. Semantic word pairs are those linked by their shared features
and meaning, such as \emph{wasp} and \emph{bee} ({\textbf{???}},
{\textbf{???}}, {\textbf{???}}).

Associative and semantic relationships between words are experimentally
definable by the use of normed databases. ({\textbf{???}}) took the
online dictionary, WordNet ({\textbf{???}}), and used software by
({\textbf{???}}) to create a database of words displaying the semantic
distance between individual words. This database displays the
relatedness between two words by measuring how semantically close words
appear in hierarchy, or the JCN ({\textbf{???}}). JCN measures the word
pairs' informational distance from one another, or their semantic
similarities. Therefore, a low JCN score demonstrates a close semantic
relationship. Additionally, we can use a measure of semantic feature
overlap to examine the semantic relatedness between word pairs
({\textbf{???}}, {\textbf{???}}, {\textbf{???}}), and this measure is
factorally related to JCN as a semantic measure ({\textbf{???}}).
Another useful database, created by ({\textbf{???}}), is centered on the
associative relationships between words. Participants were given cue
words and asked to write the first word that came to mind. These
responses were asked of and averaged over many participants. The
probability of a cue word eliciting the target word is called the
forward strength (FSG). For example, when participants are shown the
word \emph{lost}, the most common response is \emph{found}, which has a
FSG of .75 or occurs about 75\% of the time.

\subsection{Separating Semantic and Associative
Priming}\label{separating-semantic-and-associative-priming}

A meta-analytic review from ({\textbf{???}}) examined semantic priming
in the absence of association. Effect sizes for semantic priming alone
were lower than associative priming. However, with the addition of an
associative relationship to an existing semantic relationship, priming
effects nearly doubled, also known as the associative boost
({\textbf{???}}). This result suggests that semantic relationships, that
concurrently have associations, can increase priming effects. Priming
effects, therefore, are suggested not to be based on association in
isolation. ({\textbf{???}}) argues against Lucas, suggesting positive
evidence for associative priming. Automatic priming was sensitive to
associative strength as well as feature overlap. These points of
contention provide impetus for more research centering on distinctions
between associative and semantic priming.

With the databases described above, orthogonal word pair stimuli can be
created to examine associative and semantic priming individually and
indeed, priming can be found for each relation separately
({\textbf{???}}). Few studies have directly compared associative and
semantic relationships, especially focusing on the brain.
({\textbf{???}}) claim that hemispheric differences exist in
lexico-semantic representation, comparing associative and semantic
priming. Deacon et al. concluded that semantic features are localized in
the right hemisphere, whereas association is localized more within the
left hemisphere of the brain. The current study, with an aim to
elaborate on basic theoretical questions such as the relationship
between associative and semantic processing, examined the relationship
between N400 activation, priming task, and word relationship type.
Participants were given both a single lexical decision and letter search
task, along with separate semantic, associative, and unrelated word
pairs. We expected that the N400 modulation might vary from the
different types of word relation, which would indicate differences in c
ognitive processing and word organization.

\section{Method}\label{method}

\subsection{Participants}\label{participants}

Twenty undergraduate students were recruited from the University of
Mississippi (thirteen women and seven men), and all volunteered to
participate. All participants were English speakers. The experiment was
carried out with the permission of the University's Institutional Review
Board, and all participants signed corresponding consent forms. One
participant's data was corrupted and could not be used, and another
participant was excluded for poor task performance (below chance),
leaving eighteen participants (twelve women and six men).

\subsection{Apparatus}\label{apparatus}

The system used was a 32 Channel EEG Cap connected to a NuAmps monopolar
digital amplifier, which was connected to a computer running SCAN 4.5
software to record the data. The SCAN software was capable of managing
continuous digital data captured by the NuAmps amplifier. STIM2 was used
to coordinate the timing issues associated with Windows operating system
and collecting EEG data on a separate computer. STIM2 also served as the
software base for programming and operating experiments of this nature.
The sensors in the EEG cap were sponges injected with 130 ml of
electrically conductive solution (non-toxic and non-irritating). Also,
to protect the participants and equipment, a surge protector was used at
all times during data acquisition. The sensors recorded electrical
activity just below the scalp, displaying brain activation. This data
was amplified by the NuAmps hardware, and processed and recorded by the
SCAN software.

\subsection{Materials}\label{materials}

This experiment consisted of 360 word pairs separated into levels in
which the target words were unrelated to the prime (120), semantically
associated to the prime (60), associatively related to the prime (60),
or were nonwords (120). We used only a small number of related word
pairs to try to reduce expectancy effects described in the introduction
({\textbf{???}}). These 360 pairs were split evenly between the lexical
decision and letter search task, therefore, each task contained 60
unrelated pairs, 30 semantically related pairs, 30 associatively related
pairs, and 60 nonword pairings. The ratio of yes/no correct answers for
words and nonwords in the lexical decision task was 2:1 and 1:1 yes/no
decisions in the letter search task. Splitting the nonword pairs over
both the letter search and lexical decision task created a higher yes/no
ratio for the lexical decision task, which was controlled for by mixing
both tasks together.

The stimuli were selected from the ({\textbf{???}}) associative word
norms and ({\textbf{???}}) semantic word norms. The associative word
pairs were chosen using the criteria that they were highly associatively
related, having an FSG score greater than .5; with little or no semantic
similarities, determined by having a JCN score of greater than 20. An
example of an associative pair would be \emph{dairy-cow}. The semantic
word pairs were chosen using the criteria that they had a high semantic
relatedness shown in a JCN of 3 or less; and were not associatively
related, having an FSG of less than .01 (e.g., \emph{inn-lodge}). The
unrelated words were chosen so that they had no similarities (were
unpaired in the databases), such as \emph{blender} and \emph{compass.}
For non-word pairs, the target word had one letter changed so that it no
longer represented a real word, yet the structure was left intact to
require that the participant process the word cognitively. Essentially,
non-words were orthographically similar to its real word counterpart,
except for the change in a single letter. For example, the word
\emph{pond} can be changed to \emph{pund} to produce a non-word target.
All materials and their database values can be found at our Open Science
Foundation page: \url{https://osf.io/h5sd6/}.

\subsection{Procedure}\label{procedure}

Testing occurred in one session consisting of six blocks of acquired
data, broken up by brief rest periods. Before each participant was
measured, the system was configured to the correct settings, and the
hardware prepared. Two reference channels, which define zero voltage,
were placed on the right and left mastoid bones.

We modeled the current task after ({\textbf{???}}) lexical decision and
letter search task combination. ({\textbf{???}}) used a choice task
procedure, where the color of the target word indicated the target task.
One color denoted lexical decision with another color denoting letter
search. The lexical decision task involved participants observing a word
onscreen and deciding whether or not it was a word or non-word (such as
\emph{tortoise} and \emph{werm}). Nonrelated word pairs were created by
taking prime and target words from related pairs and randomly
rearranging them to eliminate relationships between primes and targets.
The letter search task involved participants observing a word onscreen
and deciding whether it contained a repeated letter or not (i.e.~the
repeated letters in \emph{doctor} versus no repeated letters in
\emph{nurse}). Words were presented onscreen, and would stay there until
the participant pressed the corresponding keys for yes and no.
Participant responses were time limited and truncated to 60 seconds. The
1 and 9 keys were used on the number row of the keyboard, in the
participant's lap to help eliminate muscle movement artifact in the
data.

Participants were first given instructions on how to perform the lexical
decision task, followed by 15 practice trials. Next, they were given
instructions on how to judge the letter search task, followed by 15
practice trials. Participants were then given a practice session with
both letter search and lexical decision trials mixed together. Trials
were color coded for the type of decision participants had to complete
(i.e.~letter search was green, while lexical decision was red). The
experiment made use of six sets of 60 randomly assigned word pairs for a
total of 360 trials. These trials were presented in Arial 19-point font,
and the inter-trial interval was set to two seconds to allow complete
recording of the N400 waveform. Trials were recorded in five minute
blocks, and between blocks participants were allowed to rest to prevent
fatigue. The current task differed from ({\textbf{???}}) in that
participants responded to every word (prime and targets), instead of
only targets. Therefore, there was no typical fixed stimulus onset
asynchrony (SOA) because participant responses were self-paced.

\section{Results}\label{results}

\subsection{N400 Waveform Analysis}\label{n400-waveform-analysis}

\subsubsection{Data Screening and Analysis
Plan}\label{data-screening-and-analysis-plan}

The data were cleared of artifact data using EEGLAB, an open source
MATLAB tool for processing electrophysiological data. The program
automatically scanned for and removed artifacts caused by eye-blinking.
Next, the datasets were visually inspected and any remaining corrupted
sections were removed manually. Ninety percent of the data was retained
across all trials and stimulus types after muscular artifact data were
removed. However, a loss rate of 20-30 percent is not uncommon,
especially with older EEG systems. The data were combined by task and
stimulus type exclusively for the second word in each pair. Five sites
were chosen to examine priming for nonwords, associative and semantic
word pairs based on a survey of the literature. Fz, FCz, Cz, CPz, and Pz
were used from the midline. Oz was excluded due to equipment problems
across all participants. Using MATLAB, the N400 area under the curve was
calculated for each electrode site, stimulus, and task (averaging over
trials) around 300-500 msec after stimuli presentation. A constant score
was subtracted from all EEG points to ensure all curves were below zero
for area under the curve calculations.

XXexplain something here about the outlier analysis you did? talk about
18 participants by 5 sites by x and z creates so many data points and
only four were removed, etc. XX. Data were screened for parametric
assumptions of linearity, normality, homogeneity, and homoscedasticity.
The data were slightly negatively skewed, but with the large number of
data points for each participant, the analyses below should be robust to
this skew.

To analyze this data, we used multilevel models (MLM) to control for
correlated error due to repeated measures of sites and stimulus type for
each participant ({\textbf{???}}). These models were calculated using
the \emph{nlme} package in \emph{R} ({\textbf{???}}). First, a model
with only the intercept was compared to a model with participants as a
random intercept factor. Random intercepts allow each participant to
have different average scores for areas under the curve or peak latency
(see below). If the random intercept model was better than the intercept
only model, then all forthcoming models would include participants as a
random intercept factor. Models were compared only to the previous step
and were deamed \enquote{significant} if the likelihood ratio difference
score, \(\Delta\chi^2\) was greater than to be expected given the change
in degrees of freedom between models. Therefore, the \emph{p}-values for
each \(\Delta\chi^2\) were calculated based on \(\Delta df\), and
\(\alpha\) was set to .05. The two tasks, lexical desicion and letter
search, were analyzed in separate models with the area under the curve
as the dependent variable. The independent variables included the dummy
coded site location as a control variable, followed by stimulus type
coded as a dummy variable. In this analysis, we wished to compare each
stimulus type to every other stimulus type, and therefore, we set
\(\alpha\) for these six comparisons to .05/6 = .008. The stimuli
variable was recoded to examine all pairwise comparisons.

\subsubsection{Lexical Decision Task}\label{lexical-decision-task}

\subsubsection{Letter Search Task}\label{letter-search-task}

\begin{table}[tbp]
\begin{center}
\begin{threeparttable}
\caption{\label{tab:area-table-model}Area under curve model statistics}
\begin{tabular}{lcccccc}
\toprule
Model & $df$ & AIC & BIC & $\chi^2$ & $\Delta\chi^2$ & $p$\\
\midrule
LDT Intercept & 2 & 5,653.09 & 5,660.86 & -2,824.55 & NA & NA\\
LDT Random Intercept & 3 & 5,589.29 & 5,600.94 & -2,791.65 & 65.80 & < .001\\
LDT Full & 10 & 5,509.56 & 5,548.39 & -2,744.78 & 93.73 & < .001\\
LST Intercept & 2 & 5,570.98 & 5,578.74 & -2,783.49 & NA & NA\\
LST Random Intercept & 3 & 5,494.37 & 5,506.00 & -2,744.18 & 78.62 & < .001\\
LST Full & 10 & 5,424.68 & 5,463.46 & -2,702.34 & 83.68 & < .001\\
\bottomrule
\addlinespace
\end{tabular}
\begin{tablenotes}[para]
\textit{Note.} AIC: Aikaike Information Criterion, BIC: Bayesian Information Criterion
\end{tablenotes}
\end{threeparttable}
\end{center}
\end{table}

\begin{table}[tbp]
\begin{center}
\begin{threeparttable}
\caption{\label{tab:area-table-est}Area under curve model estimates}
\begin{tabular}{lccccc}
\toprule
Task & Predictor & $b$ & $SE$ & $t$ & $p$\\
\midrule
LDT & CZ & -28.37 & 80.83 & -0.35 & .726\\
LDT & FCZ & 62.86 & 80.83 & 0.78 & .437\\
LDT & FZ & 50.87 & 80.83 & 0.63 & .530\\
LDT & PZ & 89.80 & 80.83 & 1.11 & .267\\
LDT & Unrelated - Nonword & -137.61 & 72.24 & -1.90 & .058\\
LDT & Unrelated - Semantic & 504.04 & 72.24 & 6.98 & < .001\\
LDT & Unrelated - Associative & 348.64 & 72.24 & 4.83 & < .001\\
LDT & Nonword - Semantic & 641.65 & 72.03 & 8.91 & < .001\\
LDT & Nonword - Associative & 486.26 & 72.03 & 6.75 & < .001\\
LDT & Semantic - Associatve & -155.39 & 72.03 & -2.16 & .032\\
LST & CZ & -53.33 & 74.68 & -0.71 & .476\\
LST & FCZ & -121.96 & 74.68 & -1.63 & .103\\
LST & FZ & -109.88 & 74.95 & -1.47 & .144\\
LST & PZ & -37.85 & 74.95 & -0.50 & .614\\
LST & Unrelated - Nonword & 60.96 & 67.15 & 0.91 & .365\\
LST & Unrelated - Semantic & 456.86 & 66.55 & 6.86 & < .001\\
LST & Unrelated - Associative & 489.32 & 66.55 & 7.35 & < .001\\
LST & Nonword - Semantic & 395.90 & 67.15 & 5.90 & < .001\\
LST & Nonword - Associative & 428.36 & 67.15 & 6.38 & < .001\\
LST & Semantic - Associatve & 32.46 & 66.55 & 0.49 & .626\\
\bottomrule
\addlinespace
\end{tabular}
\begin{tablenotes}[para]
\textit{Note.} The site control level was consider CPZ. Degrees of freedom are 334 for lexical decisions tasks and 332 for letter search tasks.
\end{tablenotes}
\end{threeparttable}
\end{center}
\end{table}

\subsection{N400 Peak Analysis}\label{n400-peak-analysis}

\emph{Lexical Decision Task}. After each set was processed as described
in the data processing section, a multilevel. These stimuli were then
tested with a single sample t-test comparing each processing difference
from zero. The following hypotheses were examined. First, non-word pairs
may show significantly more negative waveforms (more negative area) due
to the need to search the lexicon before a decision can be made. Second,
semantic word paris will have significantly positive values because
priming will decrease the need to search the mental lexicon. This is
more consistent with the view that the N400 indexes initial contact with
semantic memory ({\textbf{???}}). Third, associative word paris may have
significantly different values from unrelated word pairs, but a
direction is not predicted. More positive values would indicate
automatic activation similar to semantics, while more negative values
would indicate a need to search the mental lexicon. Figure 1 depicts the
N400 curves for the selected electrode sites, and Table 1 presents
t-test values for the following conclusions. Nonwords were not found to
be significantly more negative than unrelated word pairs, which may
indicate a controlled lexicon search for both types of stimuli. Both
associative and semantic N400 attenuation were found across frontal and
midline sites, while neither CPz nor associative Pz showed reduction. In
Figure 1, associative and semantic N400 waveforms are well above the
unrelated word pairs, indicating automatic priming for both types of
relatedness, even when stimuli are controlled for opposing
relationships.

\emph{Letter Search Task}. The same five sites were analyzed as the
lexical decision task. Again, data were subtracted from unrelated word
pairs averages and then compared against zero with single sample
t-tests. The following hypotheses were expected. First, ? Since task
demands require a focus at the lexical level, nonword pairings should
not show significant differences from unrelated word pairs. However, if
word processing is automatic in a letter search task ({\textbf{???}}),
then nonwords pairs may show more negative waveforms as participants
search the lexicon for the word pair. Second, ? Semantic and associative
word pairs may have significantly positive values because priming will
decrease the need to search the mental lexicon; however some research
literature indicates that letter search tasks eliminate semantic priming
({\textbf{???}}). Positive values would indicate a priming effect, which
is evidence for activation spreading automatically within the mental
lexicon. More negative or nonsignficant values would indicate processing
at the lexical, but not semantic, level. Figure 2 portrays the N400
waveforms for the letter search task, and Table 2 contains the t-test
values for the following conclusions. Although the average nonword
waveform appears to be much lower than unrelated waveform at many sites,
the variance across subjects was very large, and no significant
differences were found. This finding could indicate that ?wordness? did
not matter since participants were searching at a lexical level for
specific letters. Nearly all sites showed significant associative and
semantic attenuation for the N400 waveform, semantic Cz being the only
exception. In comparison, this result seems to suggest that letter
search does not inhibit automatic activation of word meaning and
association. The nonsignificant relationship between nonwords and
unrelated word pairs could be either statistical power or a controlled
search process, regardless of task demands.

\subsection{Task Performance}\label{task-performance}

Task data were analyzed for correctness in the lexical decision and
letter search tasks individually. Error rates were tested with a 2X4
(task by stimulus) repeated measures ANOVA. Overall, performance in the
letter search task (M=.97, SD=.02) was equal to the lexical decision
task (M=.97, SD=.02), F(1,13)=1.54, p=.24. The interaction between task
type and stimuli was also not significant F(3,39)=1.74, p=.18. The
different types of stimuli showed a difference in performance,
F(3,39)=9.85, p\textless{}.001, between nonwords (M=.94, SD=.03,
t(13)=-3.02, p=.01) and unrelated word pairs (M=.97, SD=.01); nonwords
and associative word pairs (M=.98, SD=.01, t(14)=-5.55,
p\textless{}.001); and nonwords and semantic word pairs (M=.98, SD=.02,
t(14)=-3.45, p=.01). The other stimuli comparisons were all
non-significant, and averages by task can be provided upon request.

\subsection{Reaction Time Performance}\label{reaction-time-performance}

Reaction time data were excluded for incorrect trials. Average reaction
times were calculated for each task type and stimulus. The
({\textbf{???}}) 3 standard deviation outlier trimmer procedure was used
to eliminate very long reaction times. Next, associative, semantic, and
nonword conditions were subtracted from their matching unrelated word
conditions. Figure 4 depicts the priming differences for each condition.
Each stimulus difference was analyzed with a single sample t-test
against zero to examine for priming.

\emph{Letter Search Task}. All conditions in the letter search task were
significantly primed over unrelated words pairs, while nonwords were
significantly slower than unrelated word pairs. As shown in Figure 4,
associative words pairs were almost 200 msecs faster than unrelated word
pairs, t(17) = 3.54, p \textless{} .01, and semantic word pairs were
also around 200 msecs faster than unrelated word pairs, t(17) = 6.38,
p\textless{}.01. Nonwords were significantly slower than unrelated word
pairs by about 200 msecs, t(17) = -5.18, p\textless{}.01. Given previous
research, it was slightly surprising that semantic word pairs would be
primed during a letter search task, however, the current word list has
also shown this effect in ({\textbf{???}}), and this effect matches N400
results.

\emph{Lexical Decision Task}. Priming was found for associative word
pairs in the lexical decision task, a marginal effect semantic word
pairs, and slowing for non-word pairs when compared to unrelated word
pairs. Associations were about 120 msecs faster than unrelated word
pairs, t(17) = 2.99, p\textless{}.01. Semantic word pairs were primed
approximately 85 msecs over unrelated pairs, which approached
significance, t(17) = 1.93, p=.07. Semantic priming was expected in the
lexical decision task, and this effect was most likely due to our small
sample size. Nonwords were again 200 msecs slower than unrelated word
pairs, t(17) = -5.24, p\textless{}.01.

\section{Discussion}\label{discussion}

These experiments were designed to explore the differences between N400
activation in the brain following presentation of semantic-only,
associative-only, and unrelated word pairs in priming tasks. The N400
data and reaction time data present picture of associative and semantic
priming during both lexical decision and letter search task. Because
both tasks were designed to reduce controlled processing of cue-target
relationships, these findings imply automatic activation of word
meanings and associations, even when task demands do not warrant word
activation. Nonword activation is more problematic to interpret, as N400
waveforms are not different from unrelated word pairs, but reaction time
data is much slower. These results, taken together, may illustrate a
controlled process search of the lexicon requiring the same activation
levels. When an unrelated target word is found in the lexicon,
controlled search is terminated, while searching for a nonword continues
for more time before the search is terminated. However, ({\textbf{???}})
point to potential issues with the relationship between the N400 and
automaticity. Semantic processing, ({\textbf{???}}) discuss, is possible
in the absence of attention or a dearth of awareness.

Since findings were roughly similar for associative and semantic word
pairs, we can postulate that the activation processes for these types of
word relatedness are also roughly similar. This experiment cannot
separate if the cognitive architecture is different for associations and
semantics, but that the automatic mechanisms for priming are comparable.
One limitation is that the long stimulus onset times may have allowed
for controlled processing in the reaction time data, but the consistent
N400 attenuation suggests a quick search of the lexicon similar to an
automatic activation process. Finally, differences in activation across
gender need to be explored. Although not conclusive due to sample size,
we found that male activation across stimuli was implicated in
traditional left Broca?s area, while female activation averaged to
central parietal areas. Regardless of any potential differences, the
broad sensitivity of the N400 means it can be implemented when
investigating how information is stored in the brain. The temporal lobe
has been shown to be implicated as a source from the N400, albeit occurs
in a flexible manner, varying with different classes of stimuli
({\textbf{???}}). There are sometimes dissociations between the N400 and
reaction time measures. The use of the N400 can therefore be seen as an
especially relevant dependent measure for the reason that components can
only partially be a reflection of semantic processes relating to
response latencies ({\textbf{???}}).

To date, research has focused on semantic priming and its automaticity
without many controls for associative relationships embedded in word
pairs. Certainly there is overlap between meaning and context use of
words, but these differences can be studied separately using available
databases ({\textbf{???}}). Our current study has supported findings by
({\textbf{???}}), who showed activation during letter search tasks,
along with the many studies on automatic activation during masked
priming ({\textbf{???}}, {\textbf{???}}).

Limitations do exist within these experiments. As previously mentioned a
larger sample size would increase the power coefficient of the findings.
Future studies should focus on the extent of priming in semantic word
pairs during a letter search task, which is a controversial topic within
the literature. Since our study limited relatedness to associations or
semantics, upcoming experiments could examine the interaction between
word relationship type of N400 attenuation. ({\textbf{???}}) have shown
that N400 waveform differences can be attributed to different strengths
of semantic relatedness in a linear fashion. With more exploration into
the exact priming nature of associations and semantics, we may begin to
discover their cognitive mechanisms.

\newpage

\section{References}\label{references}

\setlength{\parindent}{-0.5in} \setlength{\leftskip}{0.5in}






\end{document}
