\documentclass[english,man]{apa6}

\usepackage{amssymb,amsmath}
\usepackage{ifxetex,ifluatex}
\usepackage{fixltx2e} % provides \textsubscript
\ifnum 0\ifxetex 1\fi\ifluatex 1\fi=0 % if pdftex
  \usepackage[T1]{fontenc}
  \usepackage[utf8]{inputenc}
\else % if luatex or xelatex
  \ifxetex
    \usepackage{mathspec}
    \usepackage{xltxtra,xunicode}
  \else
    \usepackage{fontspec}
  \fi
  \defaultfontfeatures{Mapping=tex-text,Scale=MatchLowercase}
  \newcommand{\euro}{€}
\fi
% use upquote if available, for straight quotes in verbatim environments
\IfFileExists{upquote.sty}{\usepackage{upquote}}{}
% use microtype if available
\IfFileExists{microtype.sty}{\usepackage{microtype}}{}

% Table formatting
\usepackage{longtable, booktabs}
\usepackage{lscape}
% \usepackage[counterclockwise]{rotating}   % Landscape page setup for large tables
\usepackage{multirow}		% Table styling
\usepackage{tabularx}		% Control Column width
\usepackage[flushleft]{threeparttable}	% Allows for three part tables with a specified notes section
\usepackage{threeparttablex}            % Lets threeparttable work with longtable

% Create new environments so endfloat can handle them
% \newenvironment{ltable}
%   {\begin{landscape}\begin{center}\begin{threeparttable}}
%   {\end{threeparttable}\end{center}\end{landscape}}

\newenvironment{lltable}
  {\begin{landscape}\begin{center}\begin{ThreePartTable}}
  {\end{ThreePartTable}\end{center}\end{landscape}}

  \usepackage{ifthen} % Only add declarations when endfloat package is loaded
  \ifthenelse{\equal{\string man}{\string man}}{%
   \DeclareDelayedFloatFlavor{ThreePartTable}{table} % Make endfloat play with longtable
   % \DeclareDelayedFloatFlavor{ltable}{table} % Make endfloat play with lscape
   \DeclareDelayedFloatFlavor{lltable}{table} % Make endfloat play with lscape & longtable
  }{}%



% The following enables adjusting longtable caption width to table width
% Solution found at http://golatex.de/longtable-mit-caption-so-breit-wie-die-tabelle-t15767.html
\makeatletter
\newcommand\LastLTentrywidth{1em}
\newlength\longtablewidth
\setlength{\longtablewidth}{1in}
\newcommand\getlongtablewidth{%
 \begingroup
  \ifcsname LT@\roman{LT@tables}\endcsname
  \global\longtablewidth=0pt
  \renewcommand\LT@entry[2]{\global\advance\longtablewidth by ##2\relax\gdef\LastLTentrywidth{##2}}%
  \@nameuse{LT@\roman{LT@tables}}%
  \fi
\endgroup}


\ifxetex
  \usepackage[setpagesize=false, % page size defined by xetex
              unicode=false, % unicode breaks when used with xetex
              xetex]{hyperref}
\else
  \usepackage[unicode=true]{hyperref}
\fi
\hypersetup{breaklinks=true,
            pdfauthor={},
            pdftitle={The N400's 3 As: Association, Automaticity, Attenuation (and Some Semantics Too)},
            colorlinks=true,
            citecolor=blue,
            urlcolor=blue,
            linkcolor=black,
            pdfborder={0 0 0}}
\urlstyle{same}  % don't use monospace font for urls

\setlength{\parindent}{0pt}
%\setlength{\parskip}{0pt plus 0pt minus 0pt}

\setlength{\emergencystretch}{3em}  % prevent overfull lines

\ifxetex
  \usepackage{polyglossia}
  \setmainlanguage{}
\else
  \usepackage[english]{babel}
\fi

% Manuscript styling
\captionsetup{font=singlespacing,justification=justified}
\usepackage{csquotes}
\usepackage{upgreek}

 % Line numbering
  \usepackage{lineno}
  \linenumbers


\usepackage{tikz} % Variable definition to generate author note

% fix for \tightlist problem in pandoc 1.14
\providecommand{\tightlist}{%
  \setlength{\itemsep}{0pt}\setlength{\parskip}{0pt}}

% Essential manuscript parts
  \title{The N400's 3 As: Association, Automaticity, Attenuation (and Some
Semantics Too)}

  \shorttitle{N400 Association Attenuation}


  \author{Erin M. Buchanan\textsuperscript{1}, John E. Scofield\textsuperscript{2}, \& Nathan Nunley\textsuperscript{3}}

  \def\affdep{{"", "", ""}}%
  \def\affcity{{"", "", ""}}%

  \affiliation{
    \vspace{0.5cm}
          \textsuperscript{1} Missouri State University\\
          \textsuperscript{2} University of Missouri\\
          \textsuperscript{3} University of Mississippi  }

  \authornote{
    \newcounter{author}
    Erin M. Buchanan, Department of Psychology, Missouri State University;
    John E. Scofield, Department of Psychology, University of Missouri,
    Columbia, MO, 65211; Nathan Nunley, University of Mississippi, P.O. Box
    1848, University, MS, 28677.

                      Correspondence concerning this article should be addressed to Erin M. Buchanan, 901 S National, Springfield, MO, 65897. E-mail: \href{mailto:erinbuchanan@missouristate.edu}{\nolinkurl{erinbuchanan@missouristate.edu}}
                                    }


  \abstract{The N400 waveform carries insight into the nature of linguistic
processing and may shed light into the automaticity of priming word
relationships. We investigated semantic and associative word pairs in
classic lexical decision and letter search tasks to examine their
differences in cognitive processing. Normed database information was
used to create orthogonal semantic and associative word relationships to
clearly define N400 waveforms and priming for these pairs. Participants
showed N400 reduction for related word pairs, both semantic and
associative, in comparison to unrelated word pairs. This finding was
consistent across both lexical decision and letter search tasks,
indicating automatic access for both types of relatedness. Non-word
pairs showed N400 waveforms that resembled unrelated word pairs,
indicating the controlled examination of non-advantageous words.
Response latency data nearly mirrored the EEG findings. Priming was
found for semantic and associative word relationships, while non-word
pairs were generally slower than unrelated word pairs. MAKE SURE THIS
THING STILL HOLDS}
  \keywords{association, semantics, priming, N400, EEG, lexical decision, letter
search \\

    
  }





\usepackage{amsthm}
\newtheorem{theorem}{Theorem}
\newtheorem{lemma}{Lemma}
\theoremstyle{definition}
\newtheorem{definition}{Definition}
\newtheorem{corollary}{Corollary}
\newtheorem{proposition}{Proposition}
\theoremstyle{definition}
\newtheorem{example}{Example}
\theoremstyle{remark}
\newtheorem*{remark}{Remark}
\begin{document}

\maketitle

\setcounter{secnumdepth}{0}



Semantic facilitation through priming occurs when a related cue word
speeds the processing of a following target word (Meyer \& Schvaneveldt,
1971). For example, if a person is reading about a \emph{yacht} race,
the concept \emph{boat} is easier to process because of previous
activation in semantic memory. Research suggests that priming transpires
by both automatic and controlled processes (Neely, 1991; Neely, Keefe,
\& Ross, 1989). The automatic model proposes that related words are
linked in the brain due to overlapping features (Collins \& Loftus,
1975). Target words are activated without conscious control due to
automatic spreading activation within related cognitive networks.
Lexical and feature networks are thought to be stored separately, so
that semantic priming is the activation from the feature network feeding
back into the lexical level (Stolz \& Besner, 1996). The overlap of a
second word's semantic relatedness makes word recognition easier because
it, in essence, has already been processed. The controlled process model
proposes that people actively use cognitive strategies to connect
related words together. Neely (1991) describes both expectancy
generation and post lexical matching as ways that target word processing
may be speeded. In expectancy generation, people consciously attempt to
predict the words and ideas that will appear next, especially in
sentences. Whereas, in post-lexical matching, people delay processing of
the second target word so that it can be compared to the cue word for
evaluation. In both cases, the target word is quickened by its
relationship to the cue word.

Traditionally, priming has been tested with a simple word or nonword
decision called a lexical decision task (Groot, 1984). Participants are
shown a cue or priming word, followed by a related or unrelated target
word for the word/nonword judgment. Priming occurs when the judgment for
the target is speeded for related pairs over unrelated pairs. Lexical
decision tasks have been criticized for their inability to distinguish
between automatic and controlled processing, so both single presentation
lexical decision tasks and masked priming manipulations have been
introduced to negate controlled processing (Ford, 1983). In a single
lexical decision task, participants assess both the cue and target word
individually so that word-cue pairing is not overt. Experimenters might
also mask or distort the cue word, so that participants do not believe
they can perceive the cue word. Even though words are garbled, word
perception occurs at a subliminal level and often facilitates the target
word with automatic activation.

\subsection{Priming in the Brain}\label{priming-in-the-brain}

Event related potentials (ERPs) are used to distinguish both the nature
and the automaticity of priming. The use of ERPs is advantageous, as
they measure brain activity through an electroencephalogram (EEG) with
good temporal resolution and are thought to be a sensitive measure of
real-time language processing (Kutas \& Federmeier, 2000). The N400 is a
negative waveform that occurs 400 milliseconds (ms) after the
participant is presented with a stimulus (Brown \& Hagoort, 1993). The
N400 has been described as a \emph{contextual integration process}, in
which meanings of words are integrated and functions, bridging together
sensory information and meaningful representations (Kutas \& Federmeier,
2000). The amplitude of the N400 is sensitive to contextual word
presentations, varying systematically with semantic processing. This
research justifies the use of the N400 as an appropriate dependent
measure for lexical decision tasks. When presented with related words,
there is an attenuation of the N400; meaning a more positive waveform
when compared to unrelated word presentation. This difference in
waveforms indicates a lessened contextual integration process because
word meanings are already activated.

Multiple theories of the N400, however, have been proposed and debated
on what explicitly the N400 indexes (see Federmeier \& Laszlo, 2009 for
a review). On one hand, processes associated with the N400 are believed
to occur post-word recognition. Brown and Hagoort (1993) examined a
lexical decision task paired with masked priming. No differences were
found in the N400 wave between related and unrelated words in the masked
prime condition. Brown and Hagoort (1993) concluded that this finding
indicated that semantic activation was a controlled process, because
attenuation only occurred when words were known. Thus, an integrating
process transpires with semantic information from of multi-word
characteristic representations (Hagoort, Baggio, \& Willems, 2009; Kutas
\& Federmeier, 2011). This condition supposedly rules out automatic
processes, because the masked prime condition only allowed automatic
processes to take place. Masked priming did not allow the participants
to consciously name the prime words they had seen; thus, they were not
able to purposefully employ conscious cognitive strategies in processing
these words. However, Deacon, Hewitt, Yang, and Nagata (2000) have found
that with shorter stimulus onset asynchronies (SOAs), this effect of
masked priming disappears. SOAs are the time interval between the prime
word presentation and the target word appearance. Short SOAs are thought
to only allow for automatic processing because the controlled, attention
based processing has not had time yet to occur. Their study displayed
masked primes long enough to enhance priming, while remaining
imperceptible. With these modifications, Deacon et al. (2000) found
equal N400 attenuation for the masked and unmasked primes. This result
would indicate that automatic activation was taking place, as the masked
prime condition did not allow controlled processes to take place. Kiefer
(2002) has found similar results in the N400 using different masking
levels, which kept judgment ability of prime words below chance.

A separate theory suggests that N400 effects are seen pre-word
recognition. The N400 was found to be sensitive to pseudo- or non-words,
even when they did not resemble their real word counterparts. Deacon,
Dynowska, Ritter, and Grose-Fifer (2004) explain that this result could
imply processes that precede word recognition, such as orthographic or
phonological analysis. Similiarly, Rolke, Heil, Streb, and Hennighausen
(2001) used the attention blink rapid serial visual presentation (RSVP)
paradigm, in which participants identified target words within a stream
of distractor words presented in a different color. They found automatic
activation of N400 semantic information even when targets were missed
(Rolke et al., 2001), suggesting pre-word recognition effects. If the
N400 indexes orthographic information, letter search tasks may cause an
N400 reduction as letter search tasks reduce semantic priming by
focusing attention on the lexical level instead of a feature meaning
level (Friedrich, Henik, \& Tzelgov, 1991). In this task, participants
are asked to determine if cue and target words contain a specific letter
presented. Stolz and Besner (1996) stipulate that this focus eliminates
or reduces priming indicates non-automatic semantic priming. However, it
is also important to note that Tse and Neely (2007) did yield evidence
that letter search primes produced semantic priming for low-frequency
targets, albeit not for high-frequency targets. In Smith and Besner
(2001) letter search and lexical decision combined study, they found
that the letter search task eliminated semantic priming when compared to
unrelated word pairs and the lexical decision task. However, Marí-Beffa,
Valdés, Cullen, Catena, and Houghton (2005) found ERP evidence for
semantic processing of the prime word during letter search tasks with
the attenuation of the N400. Regardless of competing aspects as to what
the N400 is estimated to index, vital insights have been made crossing
different cognitive domains, with the N400 illuminating aspects
originating from these different domains (Kutas \& Federmeier, 2011).

\subsection{Association}\label{association}

From a theoretical standpoint, the relation between associative and
semantic processing follows a deep line of research. Associative word
pairs are words that are linked in one's memory by contextual relation,
such as \emph{basket} and \emph{picnic} (Nelson, McEvoy, \& Schreiber,
2004). Another example would be a word pair such as \emph{alien} and
\emph{predator}, which would be associatively linked for Americans
because of the movies and popular culture. Semantic word pairs are those
linked by their shared features and meaning, such as \emph{wasp} and
\emph{bee} (Buchanan, Holmes, Teasley, \& Hutchison, 2013; McRae, Cree,
Seidenberg, \& McNorgan, 2005; Vinson \& Vigliocco, 2008).

Associative and semantic relations between words are experimentally
definable by the use of normed databases. Maki, McKinley, and Thompson
(2004) took the online dictionary, WordNet (Felbaum, 1998), and used
software by Pedersen, Patwardhan, and Michelizzi (2004) to create a
database of words displaying the semantic distance between individual
words. This database displays the relatedness between two words by
measuring how semantically close words appear in hierarchy, or the JCN
(Jiang \& Conrath, 1997). JCN measures the word pairs' informational
distance from one another, or their semantic similarities. Therefore, a
low JCN score demonstrates a close semantic relationship. Additionally,
we can use a measure of semantic feature overlap to examine the semantic
relatedness between word pairs (Buchanan et al., 2013; McRae et al.,
2005; Vinson \& Vigliocco, 2008), and this measure is factorally related
to JCN as a semantic measure (Maki \& Buchanan, 2008). Another useful
database, created by Nelson et al. (2004), is centered on the
associative relations between words. Participants were given cue words
and asked to write the first word that came to mind. These responses
were asked of and averaged over many participants. The probability of a
cue word eliciting the target word is called the forward strength (FSG).
For example, when participants are shown the word \emph{lost}, the most
common response is \emph{found}, which has a FSG of .75 or occurs about
75\% of the time.

\subsection{Separating Semantic and Associative
Priming}\label{separating-semantic-and-associative-priming}

A meta-analytic review from Lucas (2000) examined semantic priming in
the absence of association. Effect sizes for semantic priming alone were
lower than associative priming. However, with the addition of an
associative relation to an existing semantic relation, priming effects
nearly doubled, also known as the associative boost (Moss, Ostrin,
Tyler, \& Marslen-Wilson, 1995). This result suggests that semantic
relations, that concurrently have associations, can increase priming
effects. Priming effects, therefore, are suggested not to be based on
association in isolation. Hutchison (2003) argues against Lucas,
suggesting positive evidence for associative priming. Automatic priming
was sensitive to associative strength as well as semantic feature
overlap. These points of contention provide impetus for more research
centering on distinctions between associative and semantic priming
(Buchanan, 2010).

With the databases described above, orthogonal word pair stimuli can be
created to examine associative and semantic priming individually and
indeed, priming can be found for each relation separately (Buchanan,
2010). Few studies have directly compared associative and semantic
relations, especially focusing on the brain. Deacon et al. (2004) claim
that hemispheric differences exist in lexico-semantic representation,
comparing associative and semantic priming. Deacon et al. concluded that
semantic features are localized in the right hemisphere, whereas
association is localized more within the left hemisphere of the brain.
Recently, Lau, Holcomb, and Kuperberg (2013) used the Nelson et al.
(2004)'s free association norms to create highly related cue-target
pairs to explore N400 amplitudes based on predictiveness proportion;
however, this paper focuses on describing effects of semantics on N400
results, even though word-pairs were created with associative databases.
High association can and does overlap semanticity, such as
\emph{on}-\emph{off} and \emph{brother}-\emph{sister}, and this study
furthers the results from the Lau et al. (2013) by controlling for both
semantics and association in one study. Rhodes and Donaldson (2007)
expanded on previous work by Koivisto and Revonsuo (2001), wherein
semantics and association were explored separately. Both Rhodes and
Donaldson (2007) and Koivisto and Revonsuo (2001) illustrated that
association created N400 amplitude changes, while semantic relations did
not change waveforms in comparison to unrelated word pairs. The current
study expands upon this research, focusing on associative and semantic
processing, as well as examining the relationship between N400
activation and priming task. Participants were given both a single
lexical decision and letter search task, along with separate semantic,
associative, unrelated, and non-word pairs. We examined the following
hypotheses:

\begin{enumerate}
\def\labelenumi{\arabic{enumi})}
\item
  In the priming task, behavioral results showed follow with Smith and
  Besner (2001)'s previous research, although the results from separated
  association and semantic relations are unknown. We expected that
  semantic and associative pairs would show faciliation in relation
  unrelated and non-word pairs in the lexical decision task, but
  potentially no facilitation in the letter search task. This hypothesis
  was examined by using multilevel modeling regression to determine
  differences in ms response latencies across word types for each task.
\item
  The research on pre- and post-word processing for N400 changes is
  mixed, therefore, we explored N400 amplitude changes in both task
  types tentatively expecting to find changes for associative and
  semantic relations with the potential for stronger changes in the
  associative conditions (Rhodes \& Donaldson, 2007, Koivisto and
  Revonsuo (2001)). We examined this hypothesis by calculating area
  under the curve for each stimuli and using a multilevel model to
  examine differences in area for each set of peaks.
\end{enumerate}

\section{Method}\label{method}

\subsection{Participants}\label{participants}

Twenty undergraduate students were recruited from the University of
Mississippi (13 women and 7 men), and all volunteered to participate.
All participants were English speakers. The experiment was carried out
with the permission of the University?s Internal Review Board, and all
participants signed corresponding consent forms. One participant's data
was corrupted and could not be used, and another participant was
excluded for poor task performance (below chance), leaving eighteen
participants (12 women and 6 men).

\subsection{Apparatus}\label{apparatus}

The system used was a 32 Channel EEG Cap connected to a NuAmps monopolar
digital amplifier, which was connected to a computer running SCAN 4.5
software to record the data. This SCAN software was capable of managing
continuous digital data captured by the NuAmps amplifier. STIM2 was used
to coordinate the timing issues associated with Windows operating system
and collecting EEG data on a separate computer. STIM2 also served as the
software base for programming and operating experiments of this nature.
The sensors in the EEG cap were sponges injected with 130 ml of
electrically conductive solution (non-toxic and non-irritating). Also,
to protect the participants and equipment, a surge protector was used at
all times during data acquisition. The sensors recorded electrical
activity just below the scalp, displaying brain activation. This data
was amplified by the NuAmps hardware, and processed and recorded by the
SCAN software.

\subsection{Materials}\label{materials}

This experiment consisted of 360 word pairs separated into levels in
which the target words were unrelated to the prime (120), semantically
associated to the prime (60), associatively related to the prime (60),
or were nonwords (120). We used only a small number of related word
pairs to try to reduce expectancy effects described in the introduction.
These 360 pairs were split evenly between the lexical decision and
letter search task, therefore, each task contained 60 unrelated pairs,
30 semantically related pairs, 30 associatively related pairs, and 60
nonword pairings. The ratio of yes/no correct answers for words and
nonwords in the lexical decision task was 2:1 and 1:1 yes/no decisions
in the letter search task. Splitting the nonword pairs over both the
letter search and lexical decision task created a higher yes/no ratio
for the lexical decision task, which was controlled for by mixing both
tasks together.

The stimuli were selected from the Nelson et al. (2004) associative word
norms, and Maki et al. (2004) semantic word norms. The associative word
pairs were chosen using the criteria that they were highly associatively
related, having an FSG score greater than .5; with little or no semantic
similarities, determined by having a JCN score of greater than 20. An
example of an associative pair would be dairy-cow. The semantic word
pairs were chosen using the criteria that they had a high semantic
relatedness shown in a JCN of 3 or less; and were not associatively
related, having an FSG of less than .01 (e.g., inn-lodge). The unrelated
words were chosen so that they had no similarities (were unpaired in the
databases), such as blender and compass. For non-word pairs, the target
word had one letter changed so that it no longer represented a real
word, yet the structure was left intact to require that the participant
process the word cognitively. Essentially, non-words were
orthographically similar to its real word counterpart, except for the
change in a single letter (The word pond can be changed to pund to
produce a non-word target).

\subsection{Procedure}\label{procedure}

Testing occurred in one session consisting of six blocks of acquired
data, broken up by brief rest periods. Before each participant was
measured, the system was configured to the correct settings, and the
hardware prepared. Two reference channels, which define zero voltage,
were placed on the right and left mastoid bones.

We modeled the current task after Smith and Besner (2001) lexical
decision and letter search task combination. Smith and Besner used a
choice task procedure, where the color of the target word indicated the
target task. One color denoted lexical decision with another color
denoting letter search. The lexical decision task involved participants
observing a word onscreen and deciding whether or not it was a word or
non-word (such as tortoise and werm). Nonrelated word pairs were created
by taking prime and target words from related pairs and randomly
rearranging them to eliminate relationships between primes and targets.
The letter search task involved participants observing a word onscreen
and deciding whether it contained a repeated letter or not (i.e.~the
repeated letters in doctor versus no repeated letters in nurse). Words
were presented onscreen, and would stay there until the participant
pressed the corresponding keys for yes and no. Participant responses
were time limited and truncated to 60 seconds. The ?1? and ?9? keys were
used on the number row of the keyboard, in the participant?s lap to help
eliminate muscle movement artifact in the data.

Participants were first given instructions on how to perform the lexical
decision task, followed by 15 practice trials. Next, they were given
instructions on how to judge the letter search task, followed by 15
practice trials. Participants were then given a practice session with
both letter search and lexical decision trials mixed together. Trials
were color coded for the type of decision participants had to complete
(i.e.~letter search was green, while lexical decision was red). The
experiment made use of six sets of 60 randomly assigned word pairs for a
total of 360 trials. These trials were presented in Arial 19-point font,
and the inter-trial interval was set to two seconds to allow complete
recording of the N400 waveform. Trials were recorded in five minute
blocks, and between blocks participants were allowed to rest to prevent
fatigue. The current task differed from Smith and Besner (2001) in that
participants responded to every word (prime and targets), instead of
only targets. Therefore, there was no typical fixed stimulus onset
asynchrony (SOA) because participant responses were self-paced.

\section{Results}\label{results}

\subsection{EEG Data}\label{eeg-data}

\emph{N400 Waveform Analysis}. The data were cleared of artifact data
using EEGLAB, an open source MATLAB tool for processing
electrophysiological data. The program automatically scanned for and
removed artifacts caused by eye-blinking. Next, the datasets were
visually inspected and any remaining corrupted sections were removed
manually. Ninety percent of the data was retained across all trials and
stimulus types after muscular artifact data were removed. However, a
loss rate of 20-30 percent is not uncommon, especially with older EEG
systems. The data were combined by task and stimulus type exclusively
for the second word in each pair. Five sites were chosen to examine
priming for nonwords, associative and semantic word pairs based on a
survey of the literature. Fz, FCz, Cz, CPz, and Pz were used from the
midline. Oz was excluded due to equipment problems across all
participants. Using MATLAB, the N400 area under the curve was calculated
for each electrode site, stimulus, and task. The area under the curve
for the N400 ranged between 300-500 msec for participants, and average
peak latency was around 405 msec after stimulus presentation.
Additionally, we examined peak XXXXXXXX

\emph{Lexical Decision Task}. After each set was processed as described
in the data processing section, a multilevel. These stimuli were then
tested with a single sample t-test comparing each processing difference
from zero. The following hypotheses were examined. First, non-word pairs
may show significantly more negative waveforms (more negative area) due
to the need to search the lexicon before a decision can be made. Second,
semantic word paris will have significantly positive values because
priming will decrease the need to search the mental lexicon. This is
more consistent with the view that the N400 indexes initial contact with
semantic memory (Kutas \& Federmeier, 2011). Third, associative word
paris may have significantly different values from unrelated word pairs,
but a direction is not predicted. More positive values would indicate
automatic activation similar to semantics, while more negative values
would indicate a need to search the mental lexicon. Figure 1 depicts the
N400 curves for the selected electrode sites, and Table 1 presents
t-test values for the following conclusions. Nonwords were not found to
be significantly more negative than unrelated word pairs, which may
indicate a controlled lexicon search for both types of stimuli. Both
associative and semantic N400 attenuation were found across frontal and
midline sites, while neither CPz nor associative Pz showed reduction. In
Figure 1, associative and semantic N400 waveforms are well above the
unrelated word pairs, indicating automatic priming for both types of
relatedness, even when stimuli are controlled for opposing
relationships.

\emph{Letter Search Task}. The same five sites were analyzed as the
lexical decision task. Again, data were subtracted from unrelated word
pairs averages and then compared against zero with single sample
t-tests. The following hypotheses were expected. First, ? Since task
demands require a focus at the lexical level, nonword pairings should
not show significant differences from unrelated word pairs. However, if
word processing is automatic in a letter search task (Marí-Beffa et al.,
2005), then nonwords pairs may show more negative waveforms as
participants search the lexicon for the word pair. Second, ? Semantic
and associative word pairs may have significantly positive values
because priming will decrease the need to search the mental lexicon;
however some research literature indicates that letter search tasks
eliminate semantic priming (Smith \& Besner, 2001). Positive values
would indicate a priming effect, which is evidence for activation
spreading automatically within the mental lexicon. More negative or
nonsignficant values would indicate processing at the lexical, but not
semantic, level. Figure 2 portrays the N400 waveforms for the letter
search task, and Table 2 contains the t-test values for the following
conclusions. Although the average nonword waveform appears to be much
lower than unrelated waveform at many sites, the variance across
subjects was very large, and no significant differences were found. This
finding could indicate that ?wordness? did not matter since participants
were searching at a lexical level for specific letters. Nearly all sites
showed significant associative and semantic attenuation for the N400
waveform, semantic Cz being the only exception. In comparison, this
result seems to suggest that letter search does not inhibit automatic
activation of word meaning and association. The nonsignificant
relationship between nonwords and unrelated word pairs could be either
statistical power or a controlled search process, regardless of task
demands.

\emph{Channel Spectrum Differences}. Figure 3 shows the channel spectrum
map for all sites and stimuli, separated by gender. The images were
examined by both tasks and word relationship and found to have the same
picture of activation. Our sample size is fairly small for male
participants (N=6), but the different configurations for gender were
striking. Female participants, examined individually, showed varied
patterns that averaged to an overall activation in the parietal region.
Male participants all displayed large left hemisphere frontal lobe
dominance, which could be attributed to Broca?s area. We acknowledge the
limitations of our small sample and EEG mapping inadequacy, but present
these findings as an interesting avenue for future research.

\subsection{Task Performance}\label{task-performance}

Task data were analyzed for correctness in the lexical decision and
letter search tasks individually. Error rates were tested with a 2X4
(task by stimulus) repeated measures ANOVA. Overall, performance in the
letter search task (M=.97, SD=.02) was equal to the lexical decision
task (M=.97, SD=.02), F(1,13)=1.54, p=.24. The interaction between task
type and stimuli was also not significant F(3,39)=1.74, p=.18. The
different types of stimuli showed a difference in performance,
F(3,39)=9.85, p\textless{}.001, between nonwords (M=.94, SD=.03,
t(13)=-3.02, p=.01) and unrelated word pairs (M=.97, SD=.01); nonwords
and associative word pairs (M=.98, SD=.01, t(14)=-5.55,
p\textless{}.001); and nonwords and semantic word pairs (M=.98, SD=.02,
t(14)=-3.45, p=.01). The other stimuli comparisons were all
non-significant, and averages by task can be provided upon request.

\subsection{Reaction Time Performance}\label{reaction-time-performance}

Reaction time data were excluded for incorrect trials. Average reaction
times were calculated for each task type and stimulus. The Van Selst and
Jolicoeur (1994) 3 standard deviation outlier trimmer procedure was used
to eliminate very long reaction times. Next, associative, semantic, and
nonword conditions were subtracted from their matching unrelated word
conditions. Figure 4 depicts the priming differences for each condition.
Each stimulus difference was analyzed with a single sample t-test
against zero to examine for priming.

\emph{Letter Search Task}. All conditions in the letter search task were
significantly primed over unrelated words pairs, while nonwords were
significantly slower than unrelated word pairs. As shown in Figure 4,
associative words pairs were almost 200 msecs faster than unrelated word
pairs, t(17) = 3.54, p \textless{} .01, and semantic word pairs were
also around 200 msecs faster than unrelated word pairs, t(17) = 6.38,
p\textless{}.01. Nonwords were significantly slower than unrelated word
pairs by about 200 msecs, t(17) = -5.18, p\textless{}.01. Given previous
research, it was slightly surprising that semantic word pairs would be
primed during a letter search task, however, the current word list has
also shown this effect in Buchanan (2010), and this effect matches N400
results.

\emph{Lexical Decision Task}. Priming was found for associative word
pairs in the lexical decision task, a marginal effect semantic word
pairs, and slowing for non-word pairs when compared to unrelated word
pairs. Associations were about 120 msecs faster than unrelated word
pairs, t(17) = 2.99, p\textless{}.01. Semantic word pairs were primed
approximately 85 msecs over unrelated pairs, which approached
significance, t(17) = 1.93, p=.07. Semantic priming was expected in the
lexical decision task, and this effect was most likely due to our small
sample size. Nonwords were again 200 msecs slower than unrelated word
pairs, t(17) = -5.24, p\textless{}.01.

\section{Discussion}\label{discussion}

These experiments were designed to explore the differences between N400
activation in the brain following presentation of semantic-only,
associative-only, and unrelated word pairs in priming tasks. The N400
data and reaction time data present picture of associative and semantic
priming during both lexical decision and letter search task. Because
both tasks were designed to reduce controlled processing of cue-target
relationships, these findings imply automatic activation of word
meanings and associations, even when task demands do not warrant word
activation. Nonword activation is more problematic to interpret, as N400
waveforms are not different from unrelated word pairs, but reaction time
data is much slower. These results, taken together, may illustrate a
controlled process search of the lexicon requiring the same activation
levels. When an unrelated target word is found in the lexicon,
controlled search is terminated, while searching for a nonword continues
for more time before the search is terminated. However, Deacon et al.
(2000) point to potential issues with the relationship between the N400
and automaticity. Semantic processing, Deacon et al. (2000) discuss, is
possible in the absence of attention or a dearth of awareness.

Since findings were roughly similar for associative and semantic word
pairs, we can postulate that the activation processes for these types of
word relatedness are also roughly similar. This experiment cannot
separate if the cognitive architecture is different for associations and
semantics, but that the automatic mechanisms for priming are comparable.
One limitation is that the long stimulus onset times may have allowed
for controlled processing in the reaction time data, but the consistent
N400 attenuation suggests a quick search of the lexicon similar to an
automatic activation process. Finally, differences in activation across
gender need to be explored. Although not conclusive due to sample size,
we found that male activation across stimuli was implicated in
traditional left Broca?s area, while female activation averaged to
central parietal areas. Regardless of any potential differences, the
broad sensitivity of the N400 means it can be implemented when
investigating how information is stored in the brain. The temporal lobe
has been shown to be implicated as a source from the N400, albeit occurs
in a flexible manner, varying with different classes of stimuli
(Federmeier \& Laszlo, 2009). There are sometimes dissociations between
the N400 and reaction time measures. The use of the N400 can therefore
be seen as an especially relevant dependent measure for the reason that
components can only partially be a reflection of semantic processes
relating to response latencies (Kutas \& Federmeier, 2011).

To date, research has focused on semantic priming and its automaticity
without many controls for associative relationships embedded in word
pairs. Certainly there is overlap between meaning and context use of
words, but these differences can be studied separately using available
databases (Hutchison, 2003). Our current study has supported findings by
Marí-Beffa et al. (2005), who showed activation during letter search
tasks, along with the many studies on automatic activation during masked
priming (Deacon et al., 2000; Kiefer, 2002).

Limitations do exist within these experiments. As previously mentioned a
larger sample size would increase the power coefficient of the findings.
Future studies should focus on the extent of priming in semantic word
pairs during a letter search task, which is a controversial topic within
the literature. Since our study limited relatedness to associations or
semantics, upcoming experiments could examine the interaction between
word relationship type of N400 attenuation. Kreher, Holcomb, and
Kuperberg (2006) have shown that N400 waveform differences can be
attributed to different strengths of semantic relatedness in a linear
fashion. With more exploration into the exact priming nature of
associations and semantics, we may begin to discover their cognitive
mechanisms.

\newpage

\section{References}\label{references}

\setlength{\parindent}{-0.5in} \setlength{\leftskip}{0.5in}

\hypertarget{refs}{}
\hypertarget{ref-Brown1993}{}
Brown, C., \& Hagoort, P. (1993). The processing nature of the N400:
Evidence from masked priming. \emph{Journal of Cognitive Neuroscience},
\emph{5}(1), 34--44.
doi:\href{https://doi.org/10.1162/jocn.1993.5.1.34}{10.1162/jocn.1993.5.1.34}

\hypertarget{ref-Buchanan2010}{}
Buchanan, E. M. (2010). Access into Memory: Differences in Judgments and
Priming for Semantic and Associative Memory. \emph{Journal of Scientific
Psychology}, (March), 1--8.

\hypertarget{ref-Buchanan2013}{}
Buchanan, E. M., Holmes, J. L., Teasley, M. L., \& Hutchison, K. A.
(2013). English semantic word-pair norms and a searchable Web portal for
experimental stimulus creation. \emph{Behavior Research Methods},
\emph{45}(3), 746--757.
doi:\href{https://doi.org/10.3758/s13428-012-0284-z}{10.3758/s13428-012-0284-z}

\hypertarget{ref-Collins1975}{}
Collins, A. M., \& Loftus, E. F. (1975). A spreading-activation theory
of semantic processing. \emph{Psychological Review}, \emph{82}(6),
407--428.
doi:\href{https://doi.org/10.1037//0033-295X.82.6.407}{10.1037//0033-295X.82.6.407}

\hypertarget{ref-Deacon2004}{}
Deacon, D., Dynowska, A., Ritter, W., \& Grose-Fifer, J. (2004).
Repetition and semantic priming of nonwords: Implications for theories
of N400 and word recognition. \emph{Psychophysiology}, \emph{41}(1),
60--74.
doi:\href{https://doi.org/10.1111/1469-8986.00120}{10.1111/1469-8986.00120}

\hypertarget{ref-Deacon2000}{}
Deacon, D., Hewitt, S., Yang, C.-M., \& Nagata, M. (2000). Event-related
potential indices of semantic priming using masked and unmasked words:
evidence that the N400 does not reflect a post-lexical process.
\emph{Cognitive Brain Research}, \emph{9}(2), 137--146.
doi:\href{https://doi.org/10.1016/S0926-6410(99)00050-6}{10.1016/S0926-6410(99)00050-6}

\hypertarget{ref-Federmeier2009}{}
Federmeier, K. D., \& Laszlo, S. (2009). Time for meaning:
Electrophysiology provides insights into dynamics of representation and
processing in semantic memory. In B. H. Ross (Ed.), \emph{Psychology of
learning and motivation} (pp. 1--44). Burlington, MA: Academic Press.

\hypertarget{ref-Felbaum1998}{}
Felbaum, C. (1998). \emph{WordNet: An Electronic Lexical Database}. MIT
Press.

\hypertarget{ref-Ford1983}{}
Ford, M. (1983). A method for obtaining measures of local parsing
complexity throughout sentences. \emph{Journal of Verbal Learning and
Verbal Behavior}, \emph{22}(2), 203--218.
doi:\href{https://doi.org/10.1016/S0022-5371(83)90156-1}{10.1016/S0022-5371(83)90156-1}

\hypertarget{ref-Friedrich1991}{}
Friedrich, F. J., Henik, A., \& Tzelgov, J. (1991). Automatic processes
in lexical access and spreading activation. \emph{Journal of
Experimental Psychology: Human Perception and Performance},
\emph{17}(3), 792--806.
doi:\href{https://doi.org/10.1037//0096-1523.17.3.792}{10.1037//0096-1523.17.3.792}

\hypertarget{ref-DeGroot1984}{}
Groot, A. M. B. de. (1984). Primed lexical decision: Combined effects of
the proportion of related prime-target pairs and the stimulus-onset
asynchrony of prime and target. \emph{The Quarterly Journal of
Experimental Psychology Section A}, \emph{36}(2), 253--280.
doi:\href{https://doi.org/10.1080/14640748408402158}{10.1080/14640748408402158}

\hypertarget{ref-Hagoort2009}{}
Hagoort, P., Baggio, G., \& Willems, R. M. (2009). Semantic unification.
In M. S. Gazzaniga (Ed.), \emph{The cognitive neurosciences} (4th ed.,
pp. 819--836). Cambridge, MA: MIT Press.

\hypertarget{ref-Hutchison2003}{}
Hutchison, K. A. (2003). Is semantic priming due to association strength
or feature overlap? A microanalytic review. \emph{Psychonomic Bulletin
\& Review}, \emph{10}(4), 785--813.
doi:\href{https://doi.org/10.3758/BF03196544}{10.3758/BF03196544}

\hypertarget{ref-Jiang1997}{}
Jiang, J. J., \& Conrath, D. W. (1997). Semantic similarity based on
corpus statistics and lexical taxonomy. In \emph{In proceedings of
international conference research on computational linguistics (rocling
x)}. Taiwan.

\hypertarget{ref-Kiefer2002}{}
Kiefer, M. (2002). The N400 is modulated by unconsciously perceived
masked words: further evidence for an automatic spreading activation
account of N400 priming effects. \emph{Cognitive Brain Research},
\emph{13}(1), 27--39.
doi:\href{https://doi.org/10.1016/S0926-6410(01)00085-4}{10.1016/S0926-6410(01)00085-4}

\hypertarget{ref-Koivisto2001}{}
Koivisto, M., \& Revonsuo, A. (2001). Cognitive representations
underlying the N400 priming effect. \emph{Cognitive Brain Research},
\emph{12}(3), 487--490.
doi:\href{https://doi.org/10.1016/S0926-6410(01)00069-6}{10.1016/S0926-6410(01)00069-6}

\hypertarget{ref-Kreher2006}{}
Kreher, D. A., Holcomb, P. J., \& Kuperberg, G. R. (2006). An
electrophysiological investigation of indirect semantic priming.
\emph{Psychophysiology}, \emph{43}(6), 550--563.
doi:\href{https://doi.org/10.1111/j.1469-8986.2006.00460.x}{10.1111/j.1469-8986.2006.00460.x}

\hypertarget{ref-Kutas2000}{}
Kutas, M., \& Federmeier, K. D. (2000). Electrophysiology reveals
semantic memory use in language comprehension. \emph{Trends in Cognitive
Sciences}, \emph{4}(12), 463--470.
doi:\href{https://doi.org/10.1016/S1364-6613(00)01560-6}{10.1016/S1364-6613(00)01560-6}

\hypertarget{ref-Kutas2011}{}
Kutas, M., \& Federmeier, K. D. (2011). Thirty years and counting:
Finding meaning in the N400 component of the Event-Related Brain
Potential (ERP). \emph{Annual Review of Psychology}, \emph{62}(1),
621--647.
doi:\href{https://doi.org/10.1146/annurev.psych.093008.131123}{10.1146/annurev.psych.093008.131123}

\hypertarget{ref-Lau2013}{}
Lau, E. F., Holcomb, P. J., \& Kuperberg, G. R. (2013). Dissociating
N400 Effects of Prediction from Association in Single-word Contexts.
\emph{Journal of Cognitive Neuroscience}, \emph{25}(3), 484--502.
doi:\href{https://doi.org/10.1162/jocn_a_00328}{10.1162/jocn\_a\_00328}

\hypertarget{ref-Lucas2000}{}
Lucas, M. (2000). Semantic priming without association: A meta-analytic
review. \emph{Psychonomic Bulletin \& Review}, \emph{7}(4), 618--630.
doi:\href{https://doi.org/10.3758/BF03212999}{10.3758/BF03212999}

\hypertarget{ref-Maki2008}{}
Maki, W. S., \& Buchanan, E. M. (2008). Latent structure in measures of
associative, semantic, and thematic knowledge. \emph{Psychonomic
Bulletin \& Review}, \emph{15}(3), 598--603.
doi:\href{https://doi.org/10.3758/PBR.15.3.598}{10.3758/PBR.15.3.598}

\hypertarget{ref-Maki2004}{}
Maki, W. S., McKinley, L. N., \& Thompson, A. G. (2004). Semantic
distance norms computed from an electronic dictionary (WordNet).
\emph{Behavior Research Methods, Instruments, \& Computers},
\emph{36}(3), 421--431.
doi:\href{https://doi.org/10.3758/BF03195590}{10.3758/BF03195590}

\hypertarget{ref-Mari-Beffa2005}{}
Marí-Beffa, P., Valdés, B., Cullen, D. J., Catena, A., \& Houghton, G.
(2005). ERP analyses of task effects on semantic processing from words.
\emph{Cognitive Brain Research}, \emph{23}(2-3), 293--305.
doi:\href{https://doi.org/10.1016/j.cogbrainres.2004.10.016}{10.1016/j.cogbrainres.2004.10.016}

\hypertarget{ref-McRae2005}{}
McRae, K., Cree, G. S., Seidenberg, M. S., \& McNorgan, C. (2005).
Semantic feature production norms for a large set of living and
nonliving things. \emph{Behavior Research Methods}, \emph{37}(4),
547--559.
doi:\href{https://doi.org/10.3758/BF03192726}{10.3758/BF03192726}

\hypertarget{ref-Meyer1971}{}
Meyer, D. E., \& Schvaneveldt, R. W. (1971). Facilitation in recognizing
pairs of words: Evidence of a dependence between retrieval operations.
\emph{Journal of Experimental Psychology}, \emph{90}(2), 227--234.
doi:\href{https://doi.org/10.1037/h0031564}{10.1037/h0031564}

\hypertarget{ref-Moss1995}{}
Moss, H. E., Ostrin, R. K., Tyler, L. K., \& Marslen-Wilson, W. D.
(1995). Accessing different types of lexical semantic information:
Evidence from priming. \emph{Journal of Experimental Psychology:
Learning, Memory, and Cognition}, \emph{21}(4), 863--883.
doi:\href{https://doi.org/10.1037//0278-7393.21.4.863}{10.1037//0278-7393.21.4.863}

\hypertarget{ref-Neely1991}{}
Neely, J. H. (1991). \emph{Semantic priming effects in visual word
recognition: A selective review of current findings and theories}.
Hillsdale, NJ: Lawrence Erlbaum Associates, Inc.

\hypertarget{ref-Neely1989}{}
Neely, J. H., Keefe, D. E., \& Ross, K. L. (1989). Semantic priming in
the lexical decision task: Roles of prospective prime-generated
expectancies and retrospective semantic matching. \emph{Journal of
Experimental Psychology: Learning, Memory, and Cognition}, \emph{15}(6),
1003--1019.
doi:\href{https://doi.org/10.1037/0278-7393.15.6.1003}{10.1037/0278-7393.15.6.1003}

\hypertarget{ref-Nelson2004}{}
Nelson, D. L., McEvoy, C. L., \& Schreiber, T. A. (2004). The University
of South Florida free association, rhyme, and word fragment norms.
\emph{Behavior Research Methods, Instruments, \& Computers},
\emph{36}(3), 402--407.
doi:\href{https://doi.org/10.3758/BF03195588}{10.3758/BF03195588}

\hypertarget{ref-Pedersen2004}{}
Pedersen, T., Patwardhan, S., \& Michelizzi, J. (2004).
WordNet::Similarity. In \emph{Demonstration papers at hlt-naacl 2004 on
xx - hlt-naacl 2004} (pp. 38--41). Morristown, NJ, USA: Association for
Computational Linguistics.
doi:\href{https://doi.org/10.3115/1614025.1614037}{10.3115/1614025.1614037}

\hypertarget{ref-Rhodes2007}{}
Rhodes, S. M., \& Donaldson, D. I. (2007). Association and not semantic
relationships elicit the N400 effect: Electrophysiological evidence from
an explicit language comprehension task. \emph{Psychophysiology},
\emph{0}(0), 070914092401002--???
doi:\href{https://doi.org/10.1111/j.1469-8986.2007.00598.x}{10.1111/j.1469-8986.2007.00598.x}

\hypertarget{ref-Rolke2001}{}
Rolke, B., Heil, M., Streb, J., \& Hennighausen, E. (2001). Missed prime
words within the attentional blink evoke an N400 semantic priming
effect. \emph{Psychophysiology}, \emph{38}(2), 165--174.
doi:\href{https://doi.org/10.1111/1469-8986.3820165}{10.1111/1469-8986.3820165}

\hypertarget{ref-Smith2001}{}
Smith, M. C., \& Besner, D. (2001). Modulating semantic feedback in
visual word recognition. \emph{Psychonomic Bulletin \& Review},
\emph{8}(1), 111--117.
doi:\href{https://doi.org/10.3758/BF03196146}{10.3758/BF03196146}

\hypertarget{ref-Stolz1996a}{}
Stolz, J. A., \& Besner, D. (1996). Role of set in visual word
recognition: Activation and activation blocking as nonautomatic
processes. \emph{Journal of Experimental Psychology: Human Perception
and Performance}, \emph{22}(5), 1166--1177.
doi:\href{https://doi.org/10.1037//0096-1523.22.5.1166}{10.1037//0096-1523.22.5.1166}

\hypertarget{ref-Tse2007}{}
Tse, C.-S., \& Neely, J. H. (2007). Semantic priming from
letter-searched primes occurs for low- but not high-frequency targets:
Automatic semantic access may not be a myth. \emph{Journal of
Experimental Psychology: Learning, Memory, and Cognition}, \emph{33}(6),
1143--1161.
doi:\href{https://doi.org/10.1037/0278-7393.33.6.1143}{10.1037/0278-7393.33.6.1143}

\hypertarget{ref-VanSelst1994}{}
Van Selst, M., \& Jolicoeur, P. (1994). Can mental rotation occur before
the dual-task bottleneck? \emph{Journal of Experimental Psychology:
Human Perception and Performance}, \emph{20}(4), 905--921.
doi:\href{https://doi.org/10.1037/0096-1523.20.4.905}{10.1037/0096-1523.20.4.905}

\hypertarget{ref-Vinson2008}{}
Vinson, D. P., \& Vigliocco, G. (2008). Semantic feature production
norms for a large set of objects and events. \emph{Behavior Research
Methods}, \emph{40}(1), 183--190.
doi:\href{https://doi.org/10.3758/BRM.40.1.183}{10.3758/BRM.40.1.183}






\end{document}
